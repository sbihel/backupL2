%%!TEX encoding = UTF-8 Unicode
\documentclass[a4paper, 11pt]{article}
\usepackage[utf8]{inputenc} 
\usepackage[T1]{fontenc}
\usepackage{graphicx}
\usepackage[french]{babel}
\usepackage{multirow,multicol}
\usepackage{amsmath, amssymb, latexsym}
\usepackage{pstricks,pst-node,pst-coil,pst-grad,pst-plot}
\usepackage{epsfig,subfigure}
\usepackage[lined,boxed]{algorithm}
\usepackage{algorithmic}
\usepackage{tikz}
\usepackage{lscape}

\usepackage{rotate}
\usepackage{url}

\usepackage{setspace}


\title{FTL : Fit the fittest©}
\author{Sébastien \textsc{Gamblin} \and  Simon \textsc{Bihel} \and Julien \textsc{Pezant} \and  Paul \textsc{Leménager} \and  Josselin \textsc{Gueneron}}
\date{}

\begin{document}

\maketitle

\tableofcontents


\begin{landscape}

\section{Constitution d'un vaisseau (table)}

%:-+-+-+- Engendré par : http://math.et.info.free.fr/TikZ/Arbre/
\begin{center}
% Racine en Haut, développement vers le bas
\begin{tikzpicture}[xscale=1,yscale=1]
% Styles (MODIFIABLES)
\tikzstyle{fleche}=[->,>=latex,thick]
\tikzstyle{noeud}=[fill=green,rectangle,draw]
\tikzstyle{feuille}=[fill=yellow,rectangle,draw]
% Dimensions (MODIFIABLES)
\def\DistanceInterNiveaux{3}
\def\DistanceInterFeuilles{2}
% Dimensions calculées (NON MODIFIABLES)
\def\NiveauA{(-0)*\DistanceInterNiveaux}
\def\NiveauB{(-1)*\DistanceInterNiveaux}
\def\NiveauC{(-2)*\DistanceInterNiveaux}
\def\InterFeuilles{(1)*\DistanceInterFeuilles}
% Noeuds (MODIFIABLES : Styles et Coefficients d'InterFeuilles)
\node[noeud] (R) at ({(5.5)*\InterFeuilles},{\NiveauA}) {$Vaisseau$};
\node[noeud] (Ra) at ({(2)*\InterFeuilles},{\NiveauB}) {$Ressources$};
\node[feuille] (Raa) at ({(0)*\InterFeuilles},{\NiveauC}) {$Drones parts$};
\node[feuille] (Rab) at ({(1)*\InterFeuilles},{\NiveauC}) {$Energy$};
\node[feuille] (Rac) at ({(2)*\InterFeuilles},{\NiveauC}) {$Missiles$};
\node[feuille] (Rad) at ({(3)*\InterFeuilles},{\NiveauC}) {$Equipage$};
\node[feuille] (Rae) at ({(4)*\InterFeuilles},{\NiveauC}) {$Scraps HC$};
\node[noeud] (Rb) at ({(5.5)*\InterFeuilles},{\NiveauB}) {$Eq off$};
\node[feuille] (Rba) at ({(5)*\InterFeuilles},{\NiveauC}) {$Armes (1)$};
\node[feuille] (Rbb) at ({(6)*\InterFeuilles},{\NiveauC}) {$Drones (2)$};
\node[noeud] (Rc) at ({(7.5)*\InterFeuilles},{\NiveauB}) {$Equ def$};
\node[feuille] (Rca) at ({(7)*\InterFeuilles},{\NiveauC}) {$Boucliers (3)$};
\node[feuille] (Rcb) at ({(8)*\InterFeuilles},{\NiveauC}) {$Drones (4)$};
\node[noeud] (Rd) at ({(9.5)*\InterFeuilles},{\NiveauB}) {$Salles$};
\node[feuille] (Rda) at ({(9)*\InterFeuilles},{\NiveauC}) {$Equipage$};
\node[feuille] (Rdb) at ({(10)*\InterFeuilles},{\NiveauC}) {$Systemes$};
\node[feuille] (Re) at ({(11)*\InterFeuilles},{\NiveauB}) {$Systemes$};
% Arcs (MODIFIABLES : Styles)
\draw[fleche] (R)--(Ra);
\draw[fleche] (Ra)--(Raa);
\draw[fleche] (Ra)--(Rab);
\draw[fleche] (Ra)--(Rac);
\draw[fleche] (Ra)--(Rad);
\draw[fleche] (Ra)--(Rae);
\draw[fleche] (R)--(Rb);
\draw[fleche] (Rb)--(Rba);
\draw[fleche] (Rb)--(Rbb);
\draw[fleche] (R)--(Rc);
\draw[fleche] (Rc)--(Rca);
\draw[fleche] (Rc)--(Rcb);
\draw[fleche] (R)--(Rd);
\draw[fleche] (Rd)--(Rda);
\draw[fleche] (Rd)--(Rdb);
\draw[fleche] (R)--(Re);
\end{tikzpicture}
\end{center}
%:-+-+-+-+- Fin

\begin{itemize}
\item (1) Energie + Missiles, Nécessite salle des armes
\item (2) Drones + énergie, nécessire salle des drones
\item (3) Energie, Nécessite salle des boucliers
\item (4) Drones + énergie, nécessire salle des drones
\end{itemize}

\end{landscape}

\endsection

\section{Un bon lien}

Un bon site avec plein d'informations : \url {http://ftl.wikia.com/wiki/FTL:_Faster_Than_Light_Wiki}
\endsection

\section{Constitution d'un vaisseau (Liste)}

\begin{itemize}
 \item ressources 
	\begin{itemize}
		\item drones parts
		\item energy
		\item missiles
		\item equipage
		\item (scraps hors combat)
	\end{itemize}
 \item equipements offensifs
	\begin{itemize}
		\item armes (utilise de l'energie et eventuellement des missiles) (requiert salle des armes)
		\item drones (utilise des drones parts et de l'energie) (requiert salle  des drones)
	\end{itemize}
 \item equipements deffensifs
	\begin{itemize}
		\item boucliers (utilise de l'energie) (requiert salle des boucliers)
		\item drones (utilise peut-etre des drones parts et de l'energie) (requiert salle  des drones)
	\end{itemize}
 \item salles
	\begin{itemize}
		\item equipage
		\item systemes
	\end{itemize}
 \item systemes 
\end{itemize}

\endsection

\section{Quelques informations}
\begin{multicols}{2}
\subsection{Types de vaisseaux}
\begin{itemize}
	\item Kestrel
	\item Engi
	\item Federation
	\item Zoltan
	\item Mantis
	\item Slug
	\item Rock
	\item Stealth
	\item Lanius
	\item Crystal
\end{itemize}
\medskip
\medskip
\endsubsection

\subsection{Noms des armes}
\begin{itemize}
	\item Missiles
	\item Lasers
	\item Ions
	\item Beams
	\item Bombs
	\item Flak
	\item Crystal
\end{itemize}
\endsubsection
\end{multicols}

\endsection

\section{Quelques questionnements}
\begin{itemize}
	\item Doit-on gérer l'extension ? (option finale ? )
	\item Comment gérer l'équipage, faire une IA aléatoire ?
	\item Faire des fichiers txt pour récupérer la liste des armes et les stats ?
	\item Comment matérialiser les salles d'un vaisseau ? Dico, matrice ?
\end{itemize}
\endsection

\section{Quelques pistes à suivre, étapes de conception}
\begin{itemize}
	\item Modélisation d'un vaisseau basique
	\item Génération aléatoire d'un vaisseau
	\item Générer les choix aléatoires d'armes avec une somme d'argent
	\item Générer les combats virtuels
	\item Faire des tournois de vaisseaux afin de selectionner le meilleur
\end{itemize}
\endsection

\end{document}
