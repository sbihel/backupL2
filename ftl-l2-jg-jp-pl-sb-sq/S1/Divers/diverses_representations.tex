%%!TEX encoding = UTF-8 Unicode
\documentclass[a4paper, 11pt]{article}
\usepackage[utf8]{inputenc} 
\usepackage[T1]{fontenc}
\usepackage{graphicx}
\usepackage[french]{babel}
\usepackage{multirow,multicol}
\usepackage{amsmath, amssymb, latexsym}
\usepackage{pstricks,pst-node,pst-coil,pst-grad,pst-plot}
\usepackage{epsfig,subfigure}
\usepackage[lined,boxed]{algorithm}
\usepackage{algorithmic}
\usepackage{tikz}
\usepackage{lscape}

\usepackage{rotate}
\usepackage{url}

\usepackage{setspace}


\title{FTL : Fit the fittest}
\author{Sébastien \textsc{Gamblin} \and  Simon \textsc{Bihel} \and Julien \textsc{Pezant} \and  Paul \textsc{Leménager} \and  Josselin \textsc{Gueneron}}
\date{}

\begin{document}

\maketitle

\tableofcontents

\section{Relations entre les modules}

%:-+-+-+- Engendré par : http://math.et.info.free.fr/TikZ/Arbre/
\begin{center}
	% Racine en Haut, développement vers le bas
	\begin{tikzpicture}[xscale=1,yscale=1]
		% Styles (MODIFIABLES)
		\tikzstyle{fleche}=[->,>=latex,thick]
		\tikzstyle{noeud}=[fill=yellow,circle,draw]
		\tikzstyle{feuille}=[fill=yellow,circle,draw]
		\tikzstyle{etiquette}=[midway,fill=white,draw]
		% Dimensions (MODIFIABLES)
		\def\DistanceInterNiveaux{3}
		\def\DistanceInterFeuilles{2}
		% Dimensions calculées (NON MODIFIABLES)
		\def\NiveauA{(-0)*\DistanceInterNiveaux}
		\def\NiveauB{(-1)*\DistanceInterNiveaux}
		\def\NiveauC{(-2)*\DistanceInterNiveaux}
		\def\NiveauD{(-3)*\DistanceInterNiveaux}
		\def\InterFeuilles{(1)*\DistanceInterFeuilles}
		% Noeuds (MODIFIABLES : Styles et Coefficients d'InterFeuilles)
		\node[feuille] (R) at ({(0)*\InterFeuilles},{\NiveauA}) {$Combat$};
		\node[feuille] (Ra) at ({(1)*\InterFeuilles},{\NiveauB}) {$Ships layouts$};
		\node[feuille] (Raa) at ({(3)*\InterFeuilles},{\NiveauC}) {$Vaisseau$};
		\node[feuille] (Raaa) at ({(1)*\InterFeuilles},{\NiveauD}) {$Systems$};
		\node[feuille] (Raab) at ({(5)*\InterFeuilles},{\NiveauD}) {$Weapons$};
		% Arcs (MODIFIABLES : Styles)
		\draw[fleche] (Ra)--(R);
		\draw[fleche] (Raa)--(Ra);
		\draw[fleche] (Raaa)--(Raa);
		\draw[fleche] (Raab)--(Raa);
	\end{tikzpicture}
\end{center}
%:-+-+-+-+- Fin
\medskip	
Combat : Gere les combats entre deux vaisseaux \newline
Ships layouts : Generation des vaisseaux de bases a partir d'une base de donnees\newline
Vaisseau : Module de base qui contient la majorite des classes relatives aux vaisseaux\newline
Systems : Gere les systemes \newline
Weapons : Gere et genere les armes\newline


\section{Elements constituants des armes}

Not in order and (probably) incomplete.

\begin{itemize}
	\item cost in scrap
	\item cost in ressources
	\item reload time
	\item necessary energy
	\item chance of fire
	\item chance of breach
	\item special damages on systemless rooms ?
	\item damages
	\item reactions to opponent's shields 
	\begin{itemize}
		\item does it do damages to shields ?
		\item can it go through shields ?
		\item can it work if the opponent's shields are on ?
	\end{itemize}
	\item can it do damages to opponent ship ?
	\item is it bombs ?
	\item can it stun opponent's crew ?
	\item can it disable systems ?
	\item number of shots
\end{itemize}


\section{Elements constituants des salles}

\begin{itemize}
	\item 1x2 or 2x2 room (squares of 10x10)
	\item crew in it
	\item breach
	\item fire
	\item oxygen
	\item doors
	\item x,y of top left hand corner and bottom right hand corner
\end{itemize}

\section{Elements constituants des portes}

\begin{itemize}
	\item Une porte existe ou non, il peut y avoir un trou à la porte.
	\item Une porte contient les deux salles qu'elle relie.
	\item Une porte fait 1 de largeur et 0 d'epaisseur, il faudra preciser deux points pour sa position.
	\item Une porte est soit fermee soit ouverte. Si il n'y a pas de porte on dira que la porte est toujours ouverte.
	\item Si les portes sont renforcees alors les ennemis doivent taper les portes pour passer. STATS INCONNUES
\end{itemize}

\section{Les couches de bouclier}

\begin{itemize}
	\item Lasers et asteroids enlevent une couche de bouclier si il y en a une.
	\item Ion ionise le systeme si il touche une couche de bouclier.
	\item Les couches se regenerent en fonction de l'energie mise dans le systeme. STATS INCONNUES
	\item Un membre de l'equipage augmente le temps de recharge de 10, 20 ou 29% en fonction de son niveau en boucliers.
	\item Un rayon ne peut pas endommager les couches de bouclier mais certains peuvent passer a travers.
	\item Avoir des couches de bouclier reduit les degats faits au points de vie par les eruptions solaires. STATS INCONNUES
	\item Missiles, bombes, teleporteurs, drone d'appareillage et les rayons d'artillerie ignorent les boucliers. (l'invisibilite et les drones de defense peuvent contrer certains)
\end{itemize}

\section{Les reacteurs}

How FTL dodge : http://videogamegeek.com/thread/1149948/engines-pilots-shields-vs-blasters-how-ftls-dodge
Je crois que certaines armes ne peuvent pas etre esquivees.
\begin{itemize}
	\item Plus de puissance, plus d'esquive
\end{itemize}

\end{document}
