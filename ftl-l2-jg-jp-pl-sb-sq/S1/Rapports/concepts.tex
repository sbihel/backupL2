\chapter{Concepts à maîtriser}

	Le but du premier semestre consistant, en fait, simplement à reproduire le jeu original, il y a juste quelques points à aborder. \\
	D'abord, nous avons dû extraire les informations du jeu, par exemple celles des vaisseaux de base. Pour cela, nous avons décidé d'utiliser des fichiers xml.
	Concernant la représentation des vaisseaux et des éléments qui les composent, nous avons utilisé la Programmation Orientée Objet.\\
	Pour s'assurer d'avoir une représentation correcte du jeu original, sans rien avoir oublier, nous avons construit une bible que voici.

	\section{La bible}
	
		\subsection{Base de données des infos de base tirées du jeu}
		
				\underline{\checkedbox Travail accompli} 

				\begin{itemize}
				  \item XML pour les armes
						\begin{itemize}
							\item tous les missiles sauf swarm
							\item tous les rayons
							\item tous les lasers de l’édition de base
							\item tous les ions
						\end{itemize}
					\item XML pour les vaisseaux
						\begin{itemize}
							\item Kestrel A et B
							\item Engi A
						\end{itemize}
				\end{itemize}
	
				\underline{Travail à terminer}
				\begin{itemize}
					\item XML pour les armes restantes
					\item XML pour les vaisseaux restants
				\end{itemize}
		
		\subsection{Cooldowns et dots (damages over time)}
		
				\underline{\checkedbox Travail accompli} 

				\begin{itemize}
					\item Recharge des boucliers
					\item Ions
					\item Réparation par les membres d’équipage 
					\item Dommages aux systèmes dûs aux feux
					\item Explosion des systèmes :
						\begin{itemize}
							\item Déclencheur de feu
							\item Dommages aux membres d’équipage présents 
						\end{itemize}
					\item Expansion du feu aléatoire
					\item Extinction des feux par les membres d’équipage
					\item Réparation des brèches et des systèmes par les membres d’équipage
					\item Gestion de l'oxygène
						\begin{itemize}
							\item Baisse du taux d'oxygène due aux brèches, feux, membres d'équipages de la race Lanius, système non-alimenté
							\item Expansion de l'oxygène (entre salles et portes ouvertes vers l'espace)
							\item Extinction des feux par manque d'oxygène
						\end{itemize}
				\end{itemize}
	
				\underline{ Travail à terminer} 

				\begin{itemize}
					\item Baisse du taux d'oxygène quand le système de l'oxygène est hacké
					\item Expansion du feu plus proche du jeu originel
				\end{itemize}
	

		\subsection{Gestion de l'équipage}
		
			\underline{\checkedbox Travail accompli} 
			
			\begin{itemize}
				\item Avantages aux systèmes (lorsque le personnage n'est pas en mouvement)
					\begin{itemize}
						\item Rechargement plus rapide des boucliers et des armes
						\item Augmentation de l'esquive grâce aux réacteurs et au pilotage
					\end{itemize}
				\item Gain d'expérience pour chaque "coup" final de la réparation d'un système et pour chaque tir bloqué, esquivé ou effectué (cela dépend bien sûr du système présent dans la salle)
				\item Mort d'un membre d'équipage quand il n'a plus de vie, résurection si le vaisseau a la clone bay
				\item Dégâts par manque d'air
				\item Remise à 0 des réparations des systèmes lorsqu'il n'y a pas de membre d'équipage dans leurs salles
			\end{itemize}
			
			\underline{Travail à terminer}
			
			\begin{itemize}
				\item Gestion des déplacements (travail en pause)
			\end{itemize}
			
		\subsection{Utilisation des armes et drones}
		
			\underline{\checkedbox Travail accompli}
			
				\begin{itemize}
					\item Double Dégâts au hull pour les salles sans systèmes par certaines armes
					\item Déclencheur de feux et de brèches
					\item Dégâts aux membres d'équipage présents dans la salle par certaines armes
				\end{itemize}
			
			\underline{Travail à terminer}
			
				\begin{itemize}
					\item Utilisation des armes ions
						\begin{itemize}
							\item Double "dommages" aux boucliers zoltan
							\item L'ionisation et les nouveaux vrais dommages ne s'accumulent pas
							\item Un système ionisé ne peut pas recevoir de bonus grâce aux membres d'équipage
						\end{itemize}
				\end{itemize}
				
		\subsection{Gestion de l'énergie}
		
			\underline{\checkedbox Travail accompli}
				
				\begin{itemize}
					\item Impossible d'ajouter de l'énergie si le système est ionisé
					\item Lorsqu'un système n'est plus ionisé, redonne automatiquement au système l'énergie qui a été enlevée
					\item Désactivation des armes/drones lorsque le système des armes/drones prend des dégâts
					\item Re-alimente automatiquement les armes/drones qui ont dû être désactivés (de façon rudimentaire)
					\item Les vaisseaux essayent de dépenser leur énergie
				\end{itemize}
			
		
		\subsection{Gestion particulière de certains systèmes}
		
			\underline{\checkedbox Travail accompli}
				
				\begin{itemize}
					\item L'invisibilité
						\begin{itemize}
							\item Activation lors de la gestion des cooldowns
							\item Remet à 0 le rechargement des armes et drones du vaisseau adverse
							\item Ionisation 5 fois de suite du système après utilisation
						\end{itemize}
					\item La résurection
						\begin{itemize}
							\item Création d'une copie du membre d'équipage qui est supprimé lorsque la copie est faite
							\item Réduction des niveaux de compétence
							\item Après un certain temps, ajout de l'objet dans la liste des membres d'équipage
						\end{itemize}
				\end{itemize}
			
			
