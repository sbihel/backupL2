\chapter{Réalisation}

	\section{Installation et configuration de l'application}
	
Notre projet est codé en Python 3 Pour tester notre programme il faut donc utiliser un terminal python correspondant.\\
A ce stade, notre programme est juste composé d'un moteur de combat, pour le voir fonctionner il faut donc lancer un test parmis ceux disponibles dans les fichiers tests\char`_combats.py et tournament\char`_test.py.


	\section{Tests et expérimentations réalisées}
	
Le but des tests est de voir si les combats se déroulent correctement. Les test à faire combattre des vaisseaux entre eux dont certains sont plus ou moins avantagés par rapport à leurs adversaires.\\
Exécuter de nombreux combats entre les mêmes vaisseaux de départ nous permet alors de voir à la fin le taux de victoire de chaque vaisseau. Il nous restera alors à vérifier les résultats sont cohérents avec les issues attendues.

De plus nous pouvons suivre les comportements des vaisseaux : qui tire, avec quelle arme, sur quelle salle, quels sont les répercutions. C'est le module \textit{displayShip.py} qui nous permet de visualiser toutes ces actions en affichant les états des vaisseaux dans le terminal.
	
	
	\section{Gestion du projet : organisation et répartition des tâches}
	
Nous avons dégagé 4 domaines de travail : le programme du moteur de combat, la construction d'une base de données issues du jeu original, les tests et enfin le rapport.\\
Le moteur de combat a été réalisé en partie par Simon BIHEL avec la contribution de Julien PEZANT pour la classe des armes et l'aide de Sébastien GAMBLIN pour une partie de l'affichage.\\
La construction de la base de données a été faite en collaboration par Julien PEZANT, Paul LEMENAGER et Simon BIHEL.\\
Les tests ont été faits en partie par Josselin GUENERON, Sébastien GAMBLIN et en partie par Simon BIHEL.\\
Enfin le rapport a été construit par Simon BIHEL et Sébastien GAMBLIN avec la contribution de chacun des autres membres par rapport à leurs connaissances de notre projet.
